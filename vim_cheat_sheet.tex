% Options for packages loaded elsewhere
\PassOptionsToPackage{unicode}{hyperref}
\PassOptionsToPackage{hyphens}{url}
%
\documentclass[landscape,twocolumn]{article}
\usepackage{lmodern}
\usepackage[a4paper,bindingoffset=0.2in,%
            left=0.2in,right=0.2in,top=0.2in,bottom=0.2in,%
            footskip=.25in]{geometry}\usepackage{amssymb,amsmath}
\usepackage{ifxetex,ifluatex}
\ifnum 0\ifxetex 1\fi\ifluatex 1\fi=0 % if pdftex
  \usepackage[T1]{fontenc}
  \usepackage[utf8]{inputenc}
  \usepackage{textcomp} % provide euro and other symbols
\else % if luatex or xetex
  \usepackage{unicode-math}
  \defaultfontfeatures{Scale=MatchLowercase}
  \defaultfontfeatures[\rmfamily]{Ligatures=TeX,Scale=1}
\fi
% Use upquote if available, for straight quotes in verbatim environments
\IfFileExists{upquote.sty}{\usepackage{upquote}}{}
\IfFileExists{microtype.sty}{% use microtype if available
  \usepackage[]{microtype}
  \UseMicrotypeSet[protrusion]{basicmath} % disable protrusion for tt fonts
}{}
\makeatletter
\@ifundefined{KOMAClassName}{% if non-KOMA class
  \IfFileExists{parskip.sty}{%
    \usepackage{parskip}
  }{% else
    \setlength{\parindent}{0pt}
    \setlength{\parskip}{6pt plus 2pt minus 1pt}}
}{% if KOMA class
  \KOMAoptions{parskip=half}}
\makeatother
\usepackage{xcolor}
\IfFileExists{xurl.sty}{\usepackage{xurl}}{} % add URL line breaks if available
\IfFileExists{bookmark.sty}{\usepackage{bookmark}}{\usepackage{hyperref}}
\hypersetup{
  hidelinks,
  pdfcreator={LaTeX via pandoc}}
\urlstyle{same} % disable monospaced font for URLs
\setlength{\emergencystretch}{3em} % prevent overfull lines
\providecommand{\tightlist}{%
  \setlength{\itemsep}{0pt}\setlength{\parskip}{0pt}}
\setcounter{secnumdepth}{-\maxdimen} % remove section numbering

\author{}
\date{}

\begin{document}

\hypertarget{vim-cheat-sheet}{%
\section{Vim Cheat Sheet}\label{vim-cheat-sheet}}

\begin{itemize}
\item
  \textbf{Note:} Extracted from the
  \href{https://github.com/rtorr/vim-cheat-sheet}{rtorr} repository.
\item
  Thanks to all the contributors of the
  \href{https://github.com/rtorr/vim-cheat-sheet}{rtorr} repository and
  to the open source community.
\end{itemize}

\hypertarget{global}{%
\subsection{Global}\label{global}}

\begin{itemize}
\tightlist
\item
  \textbf{:help} keyword - open help for keyword
\item
  \textbf{:saveas} file - save file as
\item
  \textbf{:close}- close current pane
\item
  \textbf{K} - open man page for word under the cursor
\end{itemize}

\hypertarget{cursor-movement}{%
\subsection{Cursor movement}\label{cursor-movement}}

\begin{itemize}
\tightlist
\item
  \textbf{h} - move cursor left
\item
  \textbf{j} - move cursor down
\item
  \textbf{k} - move cursor up
\item
  \textbf{l} - move cursor right
\item
  \textbf{H} - move to top of screen
\item
  \textbf{M} - move to middle of screen
\item
  \textbf{L} - move to bottom of screen
\item
  \textbf{w} - jump forwards to the start of a word
\item
  \textbf{W} - jump forwards to the start of a word (words can contain
  punctuation)
\item
  \textbf{e} - jump forwards to the end of a word
\item
  \textbf{E} - jump forwards to the end of a word (words can contain
  punctuation)
\item
  \textbf{b} - jump backwards to the start of a word
\item
  \textbf{B} - jump backwards to the start of a word (words can contain
  punctuation)
\item
  \textbf{\%} - move to matching character (default supported pairs:
  `()', `\{\}', `{[}{]}' - use :h matchpairs in vim for more info)
\item
  \textbf{0} - jump to the start of the line
\item
  \textbf{\^{}} - jump to the first non-blank character of the line
\item
  \textbf{\$} - jump to the end of the line
\item
  \textbf{g\_} - jump to the last non-blank character of the line
\item
  \textbf{gg} - go to the first line of the document
\item
  \textbf{G} - go to the last line of the document
\item
  \textbf{5G} - go to line 5
\item
  \textbf{fx} - jump to next occurrence of character
\item
  \textbf{tx} - jump to before next occurrence of character
\item
  \textbf{Fx} - jump to previous occurence of character
\item
  \textbf{Tx} - jump to after previous occurence of character
\item
  \textbf{;} - repeat previous f, t, F or T movement
\item
  \textbf{,} - repeat previous f, t, F or T movement, backwards
\item
  \textbf{\}} - jump to next paragraph (or function/block, when editing
  code)
\item
  \textbf{\{} - jump to previous paragraph (or function/block, when
  editing code)
\item
  \textbf{zz} - center cursor on screen
\item
  \textbf{Ctrl + e} - move screen down one line (without moving cursor)
\item
  \textbf{Ctrl + y} - move screen up one line (without moving cursor)
\item
  \textbf{Ctrl + b} - move back one full screen
\item
  \textbf{Ctrl + f} - move forward one full screen
\item
  \textbf{Ctrl + d} - move forward 1/2 a screen
\item
  \textbf{Ctrl + u} - move back 1/2 a screen
\end{itemize}

\textbf{-Tip:} Prefix a cursor movement command with a number to repeat
it. For example, 4j moves down 4 lines.

\hypertarget{insert-mode---insertingappending-text}{%
\subsection{Insert mode - inserting/appending
text}\label{insert-mode---insertingappending-text}}

\begin{itemize}
\tightlist
\item
  \textbf{i} - insert before the cursor
\item
  \textbf{I} - insert at the beginning of the line
\item
  \textbf{a} - insert (append) after the cursor
\item
  \textbf{A} - insert (append) at the end of the line
\item
  \textbf{o} - append (open) a new line below the current line
\item
  \textbf{O} - append (open) a new line above the current line
\item
  \textbf{ea} - insert (append) at the end of the word
\item
  \textbf{Esc} - exit insert mode
\end{itemize}

\hypertarget{editing}{%
\subsection{Editing}\label{editing}}

\begin{itemize}
\tightlist
\item
  \textbf{r} - replace a single character
\item
  \textbf{J} - join line below to the current one with one space in
  between
\item
  \textbf{gJ} - join line below to the current one without space in
  between
\item
  \textbf{gwip} - reflow paragraph
\item
  \textbf{cc} - change (replace) entire line
\item
  \textbf{C} - change (replace) to the end of the line
\item
  \textbf{c\$} - change (replace) to the end of the line
\item
  \textbf{ciw} - change (replace) entire word
\item
  \textbf{cw} - change (replace) to the end of the word
\item
  \textbf{s} - delete character and substitute text
\item
  \textbf{S} - delete line and substitute text (same as cc)
\item
  \textbf{xp} - transpose two letters (delete and paste)
\item
  \textbf{u} - undo
\item
  \textbf{Ctrl + r} - redo
\item
  \textbf{.} - repeat last command
\end{itemize}

\hypertarget{marking-text-visual-mode}{%
\subsection{Marking text (visual mode)}\label{marking-text-visual-mode}}

\begin{itemize}
\tightlist
\item
  \textbf{v} - start visual mode, mark lines, then do a command (like
  y-yank)
\item
  \textbf{V} - start linewise visual mode
\item
  \textbf{o} - move to other end of marked area
\item
  \textbf{Ctrl + v} - start visual block mode
\item
  \textbf{O} - move to other corner of block
\item
  \textbf{aw} - mark a word
\item
  \textbf{ab} - a block with ()
\item
  \textbf{aB} - a block with \{\}
\item
  \textbf{ib} - inner block with ()
\item
  \textbf{iB} - inner block with \{\}
\item
  \textbf{Esc} - exit visual mode
\end{itemize}

\hypertarget{visual-commands}{%
\subsection{Visual commands}\label{visual-commands}}

\begin{itemize}
\tightlist
\item
  \textbf{\textgreater{}} - shift text right
\item
  \textbf{\textless{}} - shift text left
\item
  \textbf{y} - yank (copy) marked text
\item
  \textbf{d} - delete marked text
\item
  \textbf{\textasciitilde{}} - switch case
\end{itemize}

\hypertarget{registers}{%
\subsection{Registers}\label{registers}}

\begin{itemize}
\tightlist
\item
  \textbf{:reg} - show registers content
\item
  \textbf{"xy} - yank into register
\item
  \textbf{"xp} - paste contents of register
\end{itemize}

\textbf{Tip:} Registers are being stored in \textasciitilde/.viminfo,
and will be loaded again on next restart of vim. Tip Register 0 contains
always the value of the last yank command. \#\# Marks

\begin{itemize}
\tightlist
\item
  \textbf{:marks} - list of marks
\item
  \textbf{ma} - set current position for mark A
\item
  \textbf{`a} - jump to position of mark A
\item
  \textbf{y`a} - yank text to position of mark A
\end{itemize}

\hypertarget{macros}{%
\subsection{Macros}\label{macros}}

\begin{itemize}
\tightlist
\item
  \textbf{qa} - record macro a
\item
  \textbf{q} - stop recording macro
\item
  \textbf{@a} - run macro a
\item
  **@@** - rerun last run macro
\end{itemize}

\hypertarget{cut-and-paste}{%
\subsection{Cut and paste}\label{cut-and-paste}}

\begin{itemize}
\tightlist
\item
  \textbf{yy} - yank (copy) a line
\item
  \textbf{2yy} - yank (copy) 2 lines
\item
  \textbf{yw} - yank (copy) the characters of the word from the cursor
  position to the start of the next word
\item
  \textbf{y\$} - yank (copy) to end of line
\item
  \textbf{p} - put (paste) the clipboard after cursor
\item
  \textbf{P} - put (paste) before cursor
\item
  \textbf{dd} - delete (cut) a line
\item
  \textbf{2dd} - delete (cut) 2 lines
\item
  \textbf{dw} - delete (cut) the characters of the word from the cursor
  position to the start of the next word
\item
  \textbf{D} - delete (cut) to the end of the line
\item
  \textbf{d\$} - delete (cut) to the end of the line
\item
  \textbf{x} - delete (cut) character
\end{itemize}

\hypertarget{exiting} - write out the current file using sudo
\item
  \textbf{:wq} or \textbf{:x} or \textbf{ZZ} - write (save) and quit
\item
  \textbf{:q} - quit (fails if there are unsaved changes)
\item
  \textbf{:q!} or \textbf{ZQ} - quit and throw away unsaved changes
\item
  \textbf{:wqa} - write (save) and quit on all tabs
\end{itemize}

\hypertarget{search-and-replace}{%
\subsection{Search and replace}\label{search-and-replace}}

\begin{itemize}
\tightlist
\item
  \textbf{/pattern} - search for pattern
\item
  \textbf{?pattern} - search backward for pattern
\item
    \textbf{\textbackslash vpattern} - `very magic' pattern: non-alphanumeric characters are
  interpreted as special regex symbols (no escaping needed)
\item
  \textbf{n} - repeat search in same direction
\item
  \textbf{N} - repeat search in opposite direction
\item
  \textbf{:\%s/old/new/g} - replace all old with new throughout file
\item
  \textbf{:\%s/old/new/gc} - replace all old with new throughout file
  with confirmations
\item
  \textbf{:noh} - remove highlighting of search matches
\end{itemize}

\hypertarget{search-in-multiple-files}{%
\subsection{Search in multiple files}\label{search-in-multiple-files}}

\begin{itemize}
\item
  \textbf{:vimgrep /pattern/ \{\texttt{\{file\}}\}} - search for pattern
  in multiple files

  \textbf{e.g.~:} vimgrep /foo/ **/*
\item
  \textbf{:cn} - jump to the next match
\item
  \textbf{:cp} - jump to the previous match
\item
  \textbf{:copen} - open a window containing the list of matches
\end{itemize}

\hypertarget{working-with-multiple-files}{%
\subsection{Working with multiple
files}\label{working-with-multiple-files}}

\begin{itemize}
\tightlist
\item
  \textbf{:e file} - edit a file in a new buffer
\item
  \textbf{:bnext} or \textbf{:bn} - go to the next buffer
\item
  \textbf{:bprev} or \textbf{:bp} - go to the previous buffer
\item
  \textbf{:bd} - delete a buffer (close a file)
\item
  \textbf{:ls} - list all open buffers
\item
  \textbf{:sp file} - open a file in a new buffer and split window
\item
  \textbf{:vsp file} - open a file in a new buffer and vertically split
  window
\item
  \textbf{Ctrl + ws} - split window
\item
  \textbf{Ctrl + ww} - switch windows
\item
  \textbf{Ctrl + wq} - quit a window
\item
  \textbf{Ctrl + wv} - split window vertically
\item
  \textbf{Ctrl + wh} - move cursor to the left window (vertical split)
\item
  \textbf{Ctrl + wl} - move cursor to the right window (vertical split)
\item
  \textbf{Ctrl + wj} - move cursor to the window below (horizontal
  split)
\item
  \textbf{Ctrl + wk} - move cursor to the window above (horizontal
  split)
\end{itemize}

\end{document}
